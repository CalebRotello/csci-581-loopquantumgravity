%%%%%%%%%%%%%%%%%%%%%%%%%%%%%%%%%%%%%%%%%
% Beamer Presentation
% LaTeX Template
% Version 1.0 (10/11/12)
%
% This template has been downloaded from:
% http://www.LaTeXTemplates.com
%
% License:
% CC BY-NC-SA 3.0 (http://creativecommons.org/licenses/by-nc-sa/3.0/)
%
%%%%%%%%%%%%%%%%%%%%%%%%%%%%%%%%%%%%%%%%%

%----------------------------------------------------------------------------------------
%	PACKAGES AND THEMES
%----------------------------------------------------------------------------------------

\documentclass{beamer}

\mode<presentation> {

% The Beamer class comes with a number of default slide themes
% which change the colors and layouts of slides. Below this is a list
% of all the themes, uncomment each in turn to see what they look like.

%\usetheme{default}
%\usetheme{AnnArbor}
%\usetheme{Antibes}
%\usetheme{Bergen}
%\usetheme{Berkeley}
%\usetheme{Berlin}
%\usetheme{Boadilla}
%\usetheme{CambridgeUS}
%\usetheme{Copenhagen}
%\usetheme{Darmstadt}
%\usetheme{Dresden}
%\usetheme{Frankfurt}
%\usetheme{Goettingen}
%\usetheme{Hannover}
%\usetheme{Ilmenau}
%\usetheme{JuanLesPins}
%\usetheme{Luebeck}
\usetheme{Madrid}
%\usetheme{Malmoe}
%\usetheme{Marburg}
%\usetheme{Montpellier}
%\usetheme{PaloAlto}
%\usetheme{Pittsburgh}
%\usetheme{Rochester}
%\usetheme{Singapore}
%\usetheme{Szeged}
%\usetheme{Warsaw}

% As well as themes, the Beamer class has a number of color themes
% for any slide theme. Uncomment each of these in turn to see how it
% changes the colors of your current slide theme.

%\usecolortheme{albatross}
%\usecolortheme{beaver}
%\usecolortheme{beetle}
%\usecolortheme{crane}
%\usecolortheme{dolphin}
%\usecolortheme{dove}
%\usecolortheme{fly}
%\usecolortheme{lily}
%\usecolortheme{orchid}
%\usecolortheme{rose}
%\usecolortheme{seagull}
%\usecolortheme{seahorse}
%\usecolortheme{whale}
%\usecolortheme{wolverine}

%\setbeamertemplate{footline} % To remove the footer line in all slides uncomment this line
%\setbeamertemplate{footline}[page number] % To replace the footer line in all slides with a simple slide count uncomment this line

%\setbeamertemplate{navigation symbols}{} % To remove the navigation symbols from the bottom of all slides uncomment this line
}

\usepackage{graphicx} % Allows including images
\usepackage{booktabs} % Allows the use of \toprule, \midrule and \bottomrule in tables

%----------------------------------------------------------------------------------------
%	TITLE PAGE
%----------------------------------------------------------------------------------------

\title[QPU Project]{Simulated Spin-foam Amplitudes in Loop Quantum Gravity} % The short title appears at the bottom of every slide, the full title is only on the title page

\author{Caleb Rotello, Hakan Ayaz} % Your name
\institute[CSM] % Your institution as it will appear on the bottom of every slide, may be shorthand to save space
{
Colorado School of Mines \\ % Your institution for the title page
%\medskip
%\textit{john@smith.com} % Your email address
}
\date{\today} % Date, can be changed to a custom date

\begin{document}

\begin{frame}
\titlepage % Print the title page as the first slide
\end{frame}

%\begin{frame}
%\frametitle{Overview} % Table of contents slide, comment this block out to remove it
%\tableofcontents % Throughout your presentation, if you choose to use \section{} and \subsection{} commands, these will automatically be printed on this slide as an overview of your presentation
%\end{frame}

%----------------------------------------------------------------------------------------
%	PRESENTATION SLIDES
%----------------------------------------------------------------------------------------

%% SLIDE 1
% LQG problem, simulated spinfoam amplitudes
% Problem setup
\begin{frame}
    \frametitle{The Project}
    \begin{itemize}
        \item \textbf{LQG}\\
        Loop-quantum gravity is a theory that attempts to merge general relativity and quantum mechanics using \textbf{spacetime atoms}, which are the smallest discrete unit of space (Planck scale). 
        \bigbreak
        \item \textbf{Quantum simulation}\\
        This is approached with a quantum simulation of \textbf{spin-foam amplitudes}. These amplitudes are the dynamics of the creation and entanglement of spacetime atoms.
        \bigbreak
        \item \textbf{Circuits}\\
        Each circuit is a representation of a specific \textbf{spin-foam network}, which is a \textbf{3+1 dimensional} representation of the spacetime atom dynamics.
    \end{itemize}
\end{frame}

%% SLIDE 2 
% 
%\begin{frame}
%    
%\end{frame}

%% SLIDE 3
% Results
\begin{frame}
    \frametitle{Results}
    %On our second submission, this is the spin-foam amplitudes and percent errors we found.
    \begin{table}[h]
    \begin{centering}
            \begin{tabular}{|c|c|c|c|}
                    \hline
                    \textbf{Network} & \textbf{Result} & \textbf{Expected} & \textbf{Error} \\ \hline
                    Zero-Spin Monopole & 0.2085 & 0.25 & 16.59\% \\ \hline
                    One-Spin Monopole & 0.2540 & 0.75 & 66.14\% \\ \hline
                    Zero-Spin Dipole 1 & 0.07025 & 0.0625 & 12.39\% \\ \hline
                    Zero-Spin Dipole 2 & 0.01627 & 0.01625 & 4.127\% \\ \hline
                    %4-Simplex 1 & 9.776e-07 & 3.815e-06 & 75.37\% \\ \hline
                    %4-Simplex 2 & 4.652e-06 & 1.526e-05 & 69.51\% \\ \hline
                    %4-Simplex 3 & 2.299e-06 & 6.104e-05 & 96.23\% \\ \hline
            \end{tabular}
            \caption{Spin-foam amplitude error of networks}
    \end{centering}
    \end{table}
    \begin{itemize}
        %\item First submission had error of 16.62\% on the zero-spin dipole 2 network, and 4.127\% on the second submission. 
        \item Zero-spin dipole 2 network had error of 16.62\% on the first submission, and 4.127\% on the second submission. 
        \bigbreak
        \item Qubit malfunction in first submission, which is why only the zero-spin dipole 2 network had results to compare with the second submission.
    \end{itemize}
    
\end{frame}

%\begin{frame}
%    \frametitle{Improvements}
%    \begin{itemize}
%        \item The first project submission had an error of 16.62\% on zero-spin dipole 2, and 4.127\% on the second submission.
%        \item Results improved drastically between submissions, which shows that students can learn how to write and make non-trivial 
%        improvements to quantum circuits within a semester.
%        \item This circuit is relatively low-depth with few swaps, which makes it optimal for NISQ computing.
%    \end{itemize}
%\end{frame}


\end{document} 