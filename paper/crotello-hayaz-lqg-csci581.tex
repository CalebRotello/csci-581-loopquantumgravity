\documentclass[a4paper,11pt,aps,tightenlines,nofootinbib]{revtex4}
\usepackage{xspace}
\usepackage[english]{babel}
\usepackage{amsfonts}
\usepackage{amsmath, amsthm, amssymb, mathrsfs}
\usepackage[dvips]{graphicx}
\usepackage{pdfpages}
\usepackage{setspace}
\usepackage{indentfirst}
\usepackage[margin=2.5cm]{geometry}
\usepackage{bbm}

\usepackage{pstricks}
\usepackage{enumerate}
\usepackage{pstricks-add}
\usepackage{epsfig}
\usepackage{pst-grad}
\usepackage{pst-plot}
\usepackage{physics}

\newcommand{\pun}[1]{\textcolor{red}{\bf #1}}

\begin{document}


%\title{Quantum simulations of topologically non-homeomorphic spacetime atoms in loop quantum gravity}
\title{Quantum simulation of a space-time atom in loop quantum gravity}
\author{Caleb Rotello}
\author{Hakan Ayaz}

\begin{abstract}
\end{abstract}


\maketitle


\section{Introduction}
%\cite{ibm-qsim-qubit-of-space}
%An overview of quantum computing and accessed hardware. Introduce large number of states 
%and simulation for transitions. What loop quantum gravity is. What hardware we had access 
%to.

%Quantum computing gives an exponential speedup to problems and a dramatic increase in spatial efficiency.
%Because of the storage capabilities of the Bloch sphere, a quantum computer with 40 qubits can easily 
%manipulate a system which has $2^{40}$ states.
With quantum computers becoming more useful in recent years, many problems with large solution spaces or problem states 
are being tested.

Loop Quantum Gravity (LQG), is a theory based on the quantization of space-time, where entangled quantum tetrahedra give rise to 
space-time in a way the unifies quantum mechanics and general relativity. 




\section{Loop Quantum Gravity}
%Stronger introduction to loop quantum gravity

\subsection{Quantum Tetrahedra}
        In any $n$-dimensional space, an $n$-simplex is the shape with the fewest possible number of faces; a 
        2-simplex is a triangle, a 3-simplex is a tetrahedron, and so on to higher dimensions. One central tenet 
        of LQG is quantized space-time \textbf{CITATION - WHY tetrahedra?}. Therefore, to get discrete space in 3 dimensions
        we will choose the 3-simplex, or tetrahedron, to be our discrete unit of space. One tetrahedron is defined by the 
        equation 
        \begin{equation} 
                \vec J_1 + \vec J_2 + \vec J_3 + \vec J_4 = 0
        \end{equation}
        where $\vec J_i = (J_x,J_y,J_z)$ is the angular momentum vector of the $i$th face \cite{covariant-lqg}. We can then define 
        a quantum tetrahedron, or qubit of space, with the following \cite{qspacetime-on-qsim}, where $\theta$ and $\phi$ are 
        angles on the Bloch sphere 
        \begin{align}
                \ket{t} &= \cos(\frac{\theta}{2})\ket{0_L} + e^{i\phi}\sin(\frac{\theta}{2}) \ket{1_L}\\
                \ket{0_L} &= \frac{1}{2}(\ket{01}-\ket{10})(\ket{01}-\ket{10})\\
                \ket{1_L} &= \frac{1}{\sqrt{3}}[\ket{1100}+\ket{0011}-\frac{1}{2}(\ket{01}+\ket{10})(\ket{01}+\ket{10})]
        \end{align}
        This quantum tetrahedron is the discrete unit of 3 dimensional quantum space.

\subsection{Space-time Atom}
        The 4-simplex is a geometric object used to create discrete 3+1 dimensional space-time \cite{simplical-decomp}, so in order to 
        properly simulate LQG spinfoam amplitudes we need to create a 4-simplex with our quantum tetrahedra.
        An $n$-simplex is created by gluing $n+1$ simplices from the $n-1$ dimension; a 2 dimension triangle is created by ``gluing'' 3 lines 
        together. To create the 4-simplex, we will glue 5 quantum tetrahedra together with entanglement between faces. 

A collection of connected quantum tetrahedra gives rise to a spin-network graph, where each node in the graph is a tetrahedron and links are formed by gluing adjacent faces \cite{gluing-polyhedra}. 

\subsection{Gaussian Constraint}
%Explore the equation $0=P_G\ket{\psi}$, its solution, how we can make it unitary

\subsection{Transition Amplitudes}
%The theoretical process at work behind transition amplitude simulation.



\section{Quantum Simulation}
%What kind of quantum simulation we did.
We performed quantum simulations in order to obtain the aforementioned vertex amplitudes. Simulations 
were intended to solve the equation 
\begin{equation}
        A(B,S) = \bra{B}P_G\ket{S}
\end{equation}
where $P_G$ is the Gaussian constraint. There is no guarantee that $P_G$ is unitary, so in order 
to simulate the vertex amplitude, we must further constrain it to the projection operator $P_G = \ket{T}\bra{T}$.

\subsection{Circuit}
Template for how circuits are formed. Function of $\ket{0_L}$ state.
\subsection{Topology}
Emergent non-homeomorphisms among the spin network.

\section{Dipole Spin Network}

\section{4-Simplex Spin Network}

\section{Results and Discussion}

\vspace{0cm}
\bibliographystyle{plainnat}
\begin{thebibliography}{99}

\bibitem{ibm-qsim-qubit-of-space}
        G. Czelusta, J. Mielczarek,
        ``Quantum simulations of a qubit of space'',
        Phys. Rev. {\bf D103}, 046001 (2021)
        [arXiv:2003.13124].

\bibitem{covariant-lqg}        
        C. Rovelli, F. Vidotto,
        ``Covariant Loop Quantum Gravity: An elementary introduction to Quantum Gravity and Spinfoam Theory'',
        Cambridge Monographs on Mathematical Physics, 2014.

\bibitem{qspacetime-on-qsim}        
        K. Li, Y. Li, M. Han, et. al.,
        ``Quantum spacetime on a quantum simulator'',
        Communications Physics {\bf 2}, 122 (2019)

\bibitem{simplical-decomp}        
        S. Lawphongpanich, n.d., 
        Simplicial decompositionSimplicial Decomposition, 
        \emph{Encyclopedia of Optimization}, Boston, MA, Springer US, pp. 2375-2378

\bibitem{gluing-polyhedra}        
        B. Bayatas, E. Bianchi, N. Yokomizo,
        ``Gluing polyhedra with entanglement in loop quantum gravity'',
        Physical Review, {\bf D98}, 026001 (2018)
        

%%%% Examples of bibliography entries        
%%%\bibitem{A variables}
%%%  A. Ashtekar,
%%%  ``New variables for classical and quantum Gravity",
%%%  Phys. Rev. Lett. {\bf 57}, 2244-2247 (1986).
%%%    
%%%\bibitem{B variables}
%%%  J. F. Barbero,
%%%  ``Real Ashtekar variables for Lorentzian signature space times",
%%%  Phys. Rev. {\bf D51}, 5507-5510 (1995),
%%%  [arXiv:gr-qc/9410014].
%%%
%%%\bibitem{lqgcan1}
%%%  A. Ashtekar, J. Lewandowski,
%%%  ``Background independent quantum gravity: A status report'', 
%%%  Class. Quant. Grav. {\bf 21}  R53 (2004),
%%%  [arXiv:gr-qc/0404018].

\end{thebibliography}  

\end{document} 